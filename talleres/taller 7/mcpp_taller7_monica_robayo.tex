\documentclass[11pt]{article}

    \usepackage[breakable]{tcolorbox}
    \usepackage{parskip} % Stop auto-indenting (to mimic markdown behaviour)
    
    \usepackage{iftex}
    \ifPDFTeX
    	\usepackage[T1]{fontenc}
    	\usepackage{mathpazo}
    \else
    	\usepackage{fontspec}
    \fi

    % Basic figure setup, for now with no caption control since it's done
    % automatically by Pandoc (which extracts ![](path) syntax from Markdown).
    \usepackage{graphicx}
    % Maintain compatibility with old templates. Remove in nbconvert 6.0
    \let\Oldincludegraphics\includegraphics
    % Ensure that by default, figures have no caption (until we provide a
    % proper Figure object with a Caption API and a way to capture that
    % in the conversion process - todo).
    \usepackage{caption}
    \DeclareCaptionFormat{nocaption}{}
    \captionsetup{format=nocaption,aboveskip=0pt,belowskip=0pt}

    \usepackage[Export]{adjustbox} % Used to constrain images to a maximum size
    \adjustboxset{max size={0.9\linewidth}{0.9\paperheight}}
    \usepackage{float}
    \floatplacement{figure}{H} % forces figures to be placed at the correct location
    \usepackage{xcolor} % Allow colors to be defined
    \usepackage{enumerate} % Needed for markdown enumerations to work
    \usepackage{geometry} % Used to adjust the document margins
    \usepackage{amsmath} % Equations
    \usepackage{amssymb} % Equations
    \usepackage{textcomp} % defines textquotesingle
    % Hack from http://tex.stackexchange.com/a/47451/13684:
    \AtBeginDocument{%
        \def\PYZsq{\textquotesingle}% Upright quotes in Pygmentized code
    }
    \usepackage{upquote} % Upright quotes for verbatim code
    \usepackage{eurosym} % defines \euro
    \usepackage[mathletters]{ucs} % Extended unicode (utf-8) support
    \usepackage{fancyvrb} % verbatim replacement that allows latex
    \usepackage{grffile} % extends the file name processing of package graphics 
                         % to support a larger range
    \makeatletter % fix for grffile with XeLaTeX
    \def\Gread@@xetex#1{%
      \IfFileExists{"\Gin@base".bb}%
      {\Gread@eps{\Gin@base.bb}}%
      {\Gread@@xetex@aux#1}%
    }
    \makeatother

    % The hyperref package gives us a pdf with properly built
    % internal navigation ('pdf bookmarks' for the table of contents,
    % internal cross-reference links, web links for URLs, etc.)
    \usepackage{hyperref}
    % The default LaTeX title has an obnoxious amount of whitespace. By default,
    % titling removes some of it. It also provides customization options.
    \usepackage{titling}
    \usepackage{longtable} % longtable support required by pandoc >1.10
    \usepackage{booktabs}  % table support for pandoc > 1.12.2
    \usepackage[inline]{enumitem} % IRkernel/repr support (it uses the enumerate* environment)
    \usepackage[normalem]{ulem} % ulem is needed to support strikethroughs (\sout)
                                % normalem makes italics be italics, not underlines
    \usepackage{mathrsfs}
    

    
    % Colors for the hyperref package
    \definecolor{urlcolor}{rgb}{0,.145,.698}
    \definecolor{linkcolor}{rgb}{.71,0.21,0.01}
    \definecolor{citecolor}{rgb}{.12,.54,.11}

    % ANSI colors
    \definecolor{ansi-black}{HTML}{3E424D}
    \definecolor{ansi-black-intense}{HTML}{282C36}
    \definecolor{ansi-red}{HTML}{E75C58}
    \definecolor{ansi-red-intense}{HTML}{B22B31}
    \definecolor{ansi-green}{HTML}{00A250}
    \definecolor{ansi-green-intense}{HTML}{007427}
    \definecolor{ansi-yellow}{HTML}{DDB62B}
    \definecolor{ansi-yellow-intense}{HTML}{B27D12}
    \definecolor{ansi-blue}{HTML}{208FFB}
    \definecolor{ansi-blue-intense}{HTML}{0065CA}
    \definecolor{ansi-magenta}{HTML}{D160C4}
    \definecolor{ansi-magenta-intense}{HTML}{A03196}
    \definecolor{ansi-cyan}{HTML}{60C6C8}
    \definecolor{ansi-cyan-intense}{HTML}{258F8F}
    \definecolor{ansi-white}{HTML}{C5C1B4}
    \definecolor{ansi-white-intense}{HTML}{A1A6B2}
    \definecolor{ansi-default-inverse-fg}{HTML}{FFFFFF}
    \definecolor{ansi-default-inverse-bg}{HTML}{000000}

    % commands and environments needed by pandoc snippets
    % extracted from the output of `pandoc -s`
    \providecommand{\tightlist}{%
      \setlength{\itemsep}{0pt}\setlength{\parskip}{0pt}}
    \DefineVerbatimEnvironment{Highlighting}{Verbatim}{commandchars=\\\{\}}
    % Add ',fontsize=\small' for more characters per line
    \newenvironment{Shaded}{}{}
    \newcommand{\KeywordTok}[1]{\textcolor[rgb]{0.00,0.44,0.13}{\textbf{{#1}}}}
    \newcommand{\DataTypeTok}[1]{\textcolor[rgb]{0.56,0.13,0.00}{{#1}}}
    \newcommand{\DecValTok}[1]{\textcolor[rgb]{0.25,0.63,0.44}{{#1}}}
    \newcommand{\BaseNTok}[1]{\textcolor[rgb]{0.25,0.63,0.44}{{#1}}}
    \newcommand{\FloatTok}[1]{\textcolor[rgb]{0.25,0.63,0.44}{{#1}}}
    \newcommand{\CharTok}[1]{\textcolor[rgb]{0.25,0.44,0.63}{{#1}}}
    \newcommand{\StringTok}[1]{\textcolor[rgb]{0.25,0.44,0.63}{{#1}}}
    \newcommand{\CommentTok}[1]{\textcolor[rgb]{0.38,0.63,0.69}{\textit{{#1}}}}
    \newcommand{\OtherTok}[1]{\textcolor[rgb]{0.00,0.44,0.13}{{#1}}}
    \newcommand{\AlertTok}[1]{\textcolor[rgb]{1.00,0.00,0.00}{\textbf{{#1}}}}
    \newcommand{\FunctionTok}[1]{\textcolor[rgb]{0.02,0.16,0.49}{{#1}}}
    \newcommand{\RegionMarkerTok}[1]{{#1}}
    \newcommand{\ErrorTok}[1]{\textcolor[rgb]{1.00,0.00,0.00}{\textbf{{#1}}}}
    \newcommand{\NormalTok}[1]{{#1}}
    
    % Additional commands for more recent versions of Pandoc
    \newcommand{\ConstantTok}[1]{\textcolor[rgb]{0.53,0.00,0.00}{{#1}}}
    \newcommand{\SpecialCharTok}[1]{\textcolor[rgb]{0.25,0.44,0.63}{{#1}}}
    \newcommand{\VerbatimStringTok}[1]{\textcolor[rgb]{0.25,0.44,0.63}{{#1}}}
    \newcommand{\SpecialStringTok}[1]{\textcolor[rgb]{0.73,0.40,0.53}{{#1}}}
    \newcommand{\ImportTok}[1]{{#1}}
    \newcommand{\DocumentationTok}[1]{\textcolor[rgb]{0.73,0.13,0.13}{\textit{{#1}}}}
    \newcommand{\AnnotationTok}[1]{\textcolor[rgb]{0.38,0.63,0.69}{\textbf{\textit{{#1}}}}}
    \newcommand{\CommentVarTok}[1]{\textcolor[rgb]{0.38,0.63,0.69}{\textbf{\textit{{#1}}}}}
    \newcommand{\VariableTok}[1]{\textcolor[rgb]{0.10,0.09,0.49}{{#1}}}
    \newcommand{\ControlFlowTok}[1]{\textcolor[rgb]{0.00,0.44,0.13}{\textbf{{#1}}}}
    \newcommand{\OperatorTok}[1]{\textcolor[rgb]{0.40,0.40,0.40}{{#1}}}
    \newcommand{\BuiltInTok}[1]{{#1}}
    \newcommand{\ExtensionTok}[1]{{#1}}
    \newcommand{\PreprocessorTok}[1]{\textcolor[rgb]{0.74,0.48,0.00}{{#1}}}
    \newcommand{\AttributeTok}[1]{\textcolor[rgb]{0.49,0.56,0.16}{{#1}}}
    \newcommand{\InformationTok}[1]{\textcolor[rgb]{0.38,0.63,0.69}{\textbf{\textit{{#1}}}}}
    \newcommand{\WarningTok}[1]{\textcolor[rgb]{0.38,0.63,0.69}{\textbf{\textit{{#1}}}}}
    
    
    % Define a nice break command that doesn't care if a line doesn't already
    % exist.
    \def\br{\hspace*{\fill} \\* }
    % Math Jax compatibility definitions
    \def\gt{>}
    \def\lt{<}
    \let\Oldtex\TeX
    \let\Oldlatex\LaTeX
    \renewcommand{\TeX}{\textrm{\Oldtex}}
    \renewcommand{\LaTeX}{\textrm{\Oldlatex}}
    % Document parameters
    % Document title
    \title{mcpp\_taller7\_monica\_robayo}
    
    
    
    
    
% Pygments definitions
\makeatletter
\def\PY@reset{\let\PY@it=\relax \let\PY@bf=\relax%
    \let\PY@ul=\relax \let\PY@tc=\relax%
    \let\PY@bc=\relax \let\PY@ff=\relax}
\def\PY@tok#1{\csname PY@tok@#1\endcsname}
\def\PY@toks#1+{\ifx\relax#1\empty\else%
    \PY@tok{#1}\expandafter\PY@toks\fi}
\def\PY@do#1{\PY@bc{\PY@tc{\PY@ul{%
    \PY@it{\PY@bf{\PY@ff{#1}}}}}}}
\def\PY#1#2{\PY@reset\PY@toks#1+\relax+\PY@do{#2}}

\expandafter\def\csname PY@tok@w\endcsname{\def\PY@tc##1{\textcolor[rgb]{0.73,0.73,0.73}{##1}}}
\expandafter\def\csname PY@tok@c\endcsname{\let\PY@it=\textit\def\PY@tc##1{\textcolor[rgb]{0.25,0.50,0.50}{##1}}}
\expandafter\def\csname PY@tok@cp\endcsname{\def\PY@tc##1{\textcolor[rgb]{0.74,0.48,0.00}{##1}}}
\expandafter\def\csname PY@tok@k\endcsname{\let\PY@bf=\textbf\def\PY@tc##1{\textcolor[rgb]{0.00,0.50,0.00}{##1}}}
\expandafter\def\csname PY@tok@kp\endcsname{\def\PY@tc##1{\textcolor[rgb]{0.00,0.50,0.00}{##1}}}
\expandafter\def\csname PY@tok@kt\endcsname{\def\PY@tc##1{\textcolor[rgb]{0.69,0.00,0.25}{##1}}}
\expandafter\def\csname PY@tok@o\endcsname{\def\PY@tc##1{\textcolor[rgb]{0.40,0.40,0.40}{##1}}}
\expandafter\def\csname PY@tok@ow\endcsname{\let\PY@bf=\textbf\def\PY@tc##1{\textcolor[rgb]{0.67,0.13,1.00}{##1}}}
\expandafter\def\csname PY@tok@nb\endcsname{\def\PY@tc##1{\textcolor[rgb]{0.00,0.50,0.00}{##1}}}
\expandafter\def\csname PY@tok@nf\endcsname{\def\PY@tc##1{\textcolor[rgb]{0.00,0.00,1.00}{##1}}}
\expandafter\def\csname PY@tok@nc\endcsname{\let\PY@bf=\textbf\def\PY@tc##1{\textcolor[rgb]{0.00,0.00,1.00}{##1}}}
\expandafter\def\csname PY@tok@nn\endcsname{\let\PY@bf=\textbf\def\PY@tc##1{\textcolor[rgb]{0.00,0.00,1.00}{##1}}}
\expandafter\def\csname PY@tok@ne\endcsname{\let\PY@bf=\textbf\def\PY@tc##1{\textcolor[rgb]{0.82,0.25,0.23}{##1}}}
\expandafter\def\csname PY@tok@nv\endcsname{\def\PY@tc##1{\textcolor[rgb]{0.10,0.09,0.49}{##1}}}
\expandafter\def\csname PY@tok@no\endcsname{\def\PY@tc##1{\textcolor[rgb]{0.53,0.00,0.00}{##1}}}
\expandafter\def\csname PY@tok@nl\endcsname{\def\PY@tc##1{\textcolor[rgb]{0.63,0.63,0.00}{##1}}}
\expandafter\def\csname PY@tok@ni\endcsname{\let\PY@bf=\textbf\def\PY@tc##1{\textcolor[rgb]{0.60,0.60,0.60}{##1}}}
\expandafter\def\csname PY@tok@na\endcsname{\def\PY@tc##1{\textcolor[rgb]{0.49,0.56,0.16}{##1}}}
\expandafter\def\csname PY@tok@nt\endcsname{\let\PY@bf=\textbf\def\PY@tc##1{\textcolor[rgb]{0.00,0.50,0.00}{##1}}}
\expandafter\def\csname PY@tok@nd\endcsname{\def\PY@tc##1{\textcolor[rgb]{0.67,0.13,1.00}{##1}}}
\expandafter\def\csname PY@tok@s\endcsname{\def\PY@tc##1{\textcolor[rgb]{0.73,0.13,0.13}{##1}}}
\expandafter\def\csname PY@tok@sd\endcsname{\let\PY@it=\textit\def\PY@tc##1{\textcolor[rgb]{0.73,0.13,0.13}{##1}}}
\expandafter\def\csname PY@tok@si\endcsname{\let\PY@bf=\textbf\def\PY@tc##1{\textcolor[rgb]{0.73,0.40,0.53}{##1}}}
\expandafter\def\csname PY@tok@se\endcsname{\let\PY@bf=\textbf\def\PY@tc##1{\textcolor[rgb]{0.73,0.40,0.13}{##1}}}
\expandafter\def\csname PY@tok@sr\endcsname{\def\PY@tc##1{\textcolor[rgb]{0.73,0.40,0.53}{##1}}}
\expandafter\def\csname PY@tok@ss\endcsname{\def\PY@tc##1{\textcolor[rgb]{0.10,0.09,0.49}{##1}}}
\expandafter\def\csname PY@tok@sx\endcsname{\def\PY@tc##1{\textcolor[rgb]{0.00,0.50,0.00}{##1}}}
\expandafter\def\csname PY@tok@m\endcsname{\def\PY@tc##1{\textcolor[rgb]{0.40,0.40,0.40}{##1}}}
\expandafter\def\csname PY@tok@gh\endcsname{\let\PY@bf=\textbf\def\PY@tc##1{\textcolor[rgb]{0.00,0.00,0.50}{##1}}}
\expandafter\def\csname PY@tok@gu\endcsname{\let\PY@bf=\textbf\def\PY@tc##1{\textcolor[rgb]{0.50,0.00,0.50}{##1}}}
\expandafter\def\csname PY@tok@gd\endcsname{\def\PY@tc##1{\textcolor[rgb]{0.63,0.00,0.00}{##1}}}
\expandafter\def\csname PY@tok@gi\endcsname{\def\PY@tc##1{\textcolor[rgb]{0.00,0.63,0.00}{##1}}}
\expandafter\def\csname PY@tok@gr\endcsname{\def\PY@tc##1{\textcolor[rgb]{1.00,0.00,0.00}{##1}}}
\expandafter\def\csname PY@tok@ge\endcsname{\let\PY@it=\textit}
\expandafter\def\csname PY@tok@gs\endcsname{\let\PY@bf=\textbf}
\expandafter\def\csname PY@tok@gp\endcsname{\let\PY@bf=\textbf\def\PY@tc##1{\textcolor[rgb]{0.00,0.00,0.50}{##1}}}
\expandafter\def\csname PY@tok@go\endcsname{\def\PY@tc##1{\textcolor[rgb]{0.53,0.53,0.53}{##1}}}
\expandafter\def\csname PY@tok@gt\endcsname{\def\PY@tc##1{\textcolor[rgb]{0.00,0.27,0.87}{##1}}}
\expandafter\def\csname PY@tok@err\endcsname{\def\PY@bc##1{\setlength{\fboxsep}{0pt}\fcolorbox[rgb]{1.00,0.00,0.00}{1,1,1}{\strut ##1}}}
\expandafter\def\csname PY@tok@kc\endcsname{\let\PY@bf=\textbf\def\PY@tc##1{\textcolor[rgb]{0.00,0.50,0.00}{##1}}}
\expandafter\def\csname PY@tok@kd\endcsname{\let\PY@bf=\textbf\def\PY@tc##1{\textcolor[rgb]{0.00,0.50,0.00}{##1}}}
\expandafter\def\csname PY@tok@kn\endcsname{\let\PY@bf=\textbf\def\PY@tc##1{\textcolor[rgb]{0.00,0.50,0.00}{##1}}}
\expandafter\def\csname PY@tok@kr\endcsname{\let\PY@bf=\textbf\def\PY@tc##1{\textcolor[rgb]{0.00,0.50,0.00}{##1}}}
\expandafter\def\csname PY@tok@bp\endcsname{\def\PY@tc##1{\textcolor[rgb]{0.00,0.50,0.00}{##1}}}
\expandafter\def\csname PY@tok@fm\endcsname{\def\PY@tc##1{\textcolor[rgb]{0.00,0.00,1.00}{##1}}}
\expandafter\def\csname PY@tok@vc\endcsname{\def\PY@tc##1{\textcolor[rgb]{0.10,0.09,0.49}{##1}}}
\expandafter\def\csname PY@tok@vg\endcsname{\def\PY@tc##1{\textcolor[rgb]{0.10,0.09,0.49}{##1}}}
\expandafter\def\csname PY@tok@vi\endcsname{\def\PY@tc##1{\textcolor[rgb]{0.10,0.09,0.49}{##1}}}
\expandafter\def\csname PY@tok@vm\endcsname{\def\PY@tc##1{\textcolor[rgb]{0.10,0.09,0.49}{##1}}}
\expandafter\def\csname PY@tok@sa\endcsname{\def\PY@tc##1{\textcolor[rgb]{0.73,0.13,0.13}{##1}}}
\expandafter\def\csname PY@tok@sb\endcsname{\def\PY@tc##1{\textcolor[rgb]{0.73,0.13,0.13}{##1}}}
\expandafter\def\csname PY@tok@sc\endcsname{\def\PY@tc##1{\textcolor[rgb]{0.73,0.13,0.13}{##1}}}
\expandafter\def\csname PY@tok@dl\endcsname{\def\PY@tc##1{\textcolor[rgb]{0.73,0.13,0.13}{##1}}}
\expandafter\def\csname PY@tok@s2\endcsname{\def\PY@tc##1{\textcolor[rgb]{0.73,0.13,0.13}{##1}}}
\expandafter\def\csname PY@tok@sh\endcsname{\def\PY@tc##1{\textcolor[rgb]{0.73,0.13,0.13}{##1}}}
\expandafter\def\csname PY@tok@s1\endcsname{\def\PY@tc##1{\textcolor[rgb]{0.73,0.13,0.13}{##1}}}
\expandafter\def\csname PY@tok@mb\endcsname{\def\PY@tc##1{\textcolor[rgb]{0.40,0.40,0.40}{##1}}}
\expandafter\def\csname PY@tok@mf\endcsname{\def\PY@tc##1{\textcolor[rgb]{0.40,0.40,0.40}{##1}}}
\expandafter\def\csname PY@tok@mh\endcsname{\def\PY@tc##1{\textcolor[rgb]{0.40,0.40,0.40}{##1}}}
\expandafter\def\csname PY@tok@mi\endcsname{\def\PY@tc##1{\textcolor[rgb]{0.40,0.40,0.40}{##1}}}
\expandafter\def\csname PY@tok@il\endcsname{\def\PY@tc##1{\textcolor[rgb]{0.40,0.40,0.40}{##1}}}
\expandafter\def\csname PY@tok@mo\endcsname{\def\PY@tc##1{\textcolor[rgb]{0.40,0.40,0.40}{##1}}}
\expandafter\def\csname PY@tok@ch\endcsname{\let\PY@it=\textit\def\PY@tc##1{\textcolor[rgb]{0.25,0.50,0.50}{##1}}}
\expandafter\def\csname PY@tok@cm\endcsname{\let\PY@it=\textit\def\PY@tc##1{\textcolor[rgb]{0.25,0.50,0.50}{##1}}}
\expandafter\def\csname PY@tok@cpf\endcsname{\let\PY@it=\textit\def\PY@tc##1{\textcolor[rgb]{0.25,0.50,0.50}{##1}}}
\expandafter\def\csname PY@tok@c1\endcsname{\let\PY@it=\textit\def\PY@tc##1{\textcolor[rgb]{0.25,0.50,0.50}{##1}}}
\expandafter\def\csname PY@tok@cs\endcsname{\let\PY@it=\textit\def\PY@tc##1{\textcolor[rgb]{0.25,0.50,0.50}{##1}}}

\def\PYZbs{\char`\\}
\def\PYZus{\char`\_}
\def\PYZob{\char`\{}
\def\PYZcb{\char`\}}
\def\PYZca{\char`\^}
\def\PYZam{\char`\&}
\def\PYZlt{\char`\<}
\def\PYZgt{\char`\>}
\def\PYZsh{\char`\#}
\def\PYZpc{\char`\%}
\def\PYZdl{\char`\$}
\def\PYZhy{\char`\-}
\def\PYZsq{\char`\'}
\def\PYZdq{\char`\"}
\def\PYZti{\char`\~}
% for compatibility with earlier versions
\def\PYZat{@}
\def\PYZlb{[}
\def\PYZrb{]}
\makeatother


    % For linebreaks inside Verbatim environment from package fancyvrb. 
    \makeatletter
        \newbox\Wrappedcontinuationbox 
        \newbox\Wrappedvisiblespacebox 
        \newcommand*\Wrappedvisiblespace {\textcolor{red}{\textvisiblespace}} 
        \newcommand*\Wrappedcontinuationsymbol {\textcolor{red}{\llap{\tiny$\m@th\hookrightarrow$}}} 
        \newcommand*\Wrappedcontinuationindent {3ex } 
        \newcommand*\Wrappedafterbreak {\kern\Wrappedcontinuationindent\copy\Wrappedcontinuationbox} 
        % Take advantage of the already applied Pygments mark-up to insert 
        % potential linebreaks for TeX processing. 
        %        {, <, #, %, $, ' and ": go to next line. 
        %        _, }, ^, &, >, - and ~: stay at end of broken line. 
        % Use of \textquotesingle for straight quote. 
        \newcommand*\Wrappedbreaksatspecials {% 
            \def\PYGZus{\discretionary{\char`\_}{\Wrappedafterbreak}{\char`\_}}% 
            \def\PYGZob{\discretionary{}{\Wrappedafterbreak\char`\{}{\char`\{}}% 
            \def\PYGZcb{\discretionary{\char`\}}{\Wrappedafterbreak}{\char`\}}}% 
            \def\PYGZca{\discretionary{\char`\^}{\Wrappedafterbreak}{\char`\^}}% 
            \def\PYGZam{\discretionary{\char`\&}{\Wrappedafterbreak}{\char`\&}}% 
            \def\PYGZlt{\discretionary{}{\Wrappedafterbreak\char`\<}{\char`\<}}% 
            \def\PYGZgt{\discretionary{\char`\>}{\Wrappedafterbreak}{\char`\>}}% 
            \def\PYGZsh{\discretionary{}{\Wrappedafterbreak\char`\#}{\char`\#}}% 
            \def\PYGZpc{\discretionary{}{\Wrappedafterbreak\char`\%}{\char`\%}}% 
            \def\PYGZdl{\discretionary{}{\Wrappedafterbreak\char`\$}{\char`\$}}% 
            \def\PYGZhy{\discretionary{\char`\-}{\Wrappedafterbreak}{\char`\-}}% 
            \def\PYGZsq{\discretionary{}{\Wrappedafterbreak\textquotesingle}{\textquotesingle}}% 
            \def\PYGZdq{\discretionary{}{\Wrappedafterbreak\char`\"}{\char`\"}}% 
            \def\PYGZti{\discretionary{\char`\~}{\Wrappedafterbreak}{\char`\~}}% 
        } 
        % Some characters . , ; ? ! / are not pygmentized. 
        % This macro makes them "active" and they will insert potential linebreaks 
        \newcommand*\Wrappedbreaksatpunct {% 
            \lccode`\~`\.\lowercase{\def~}{\discretionary{\hbox{\char`\.}}{\Wrappedafterbreak}{\hbox{\char`\.}}}% 
            \lccode`\~`\,\lowercase{\def~}{\discretionary{\hbox{\char`\,}}{\Wrappedafterbreak}{\hbox{\char`\,}}}% 
            \lccode`\~`\;\lowercase{\def~}{\discretionary{\hbox{\char`\;}}{\Wrappedafterbreak}{\hbox{\char`\;}}}% 
            \lccode`\~`\:\lowercase{\def~}{\discretionary{\hbox{\char`\:}}{\Wrappedafterbreak}{\hbox{\char`\:}}}% 
            \lccode`\~`\?\lowercase{\def~}{\discretionary{\hbox{\char`\?}}{\Wrappedafterbreak}{\hbox{\char`\?}}}% 
            \lccode`\~`\!\lowercase{\def~}{\discretionary{\hbox{\char`\!}}{\Wrappedafterbreak}{\hbox{\char`\!}}}% 
            \lccode`\~`\/\lowercase{\def~}{\discretionary{\hbox{\char`\/}}{\Wrappedafterbreak}{\hbox{\char`\/}}}% 
            \catcode`\.\active
            \catcode`\,\active 
            \catcode`\;\active
            \catcode`\:\active
            \catcode`\?\active
            \catcode`\!\active
            \catcode`\/\active 
            \lccode`\~`\~ 	
        }
    \makeatother

    \let\OriginalVerbatim=\Verbatim
    \makeatletter
    \renewcommand{\Verbatim}[1][1]{%
        %\parskip\z@skip
        \sbox\Wrappedcontinuationbox {\Wrappedcontinuationsymbol}%
        \sbox\Wrappedvisiblespacebox {\FV@SetupFont\Wrappedvisiblespace}%
        \def\FancyVerbFormatLine ##1{\hsize\linewidth
            \vtop{\raggedright\hyphenpenalty\z@\exhyphenpenalty\z@
                \doublehyphendemerits\z@\finalhyphendemerits\z@
                \strut ##1\strut}%
        }%
        % If the linebreak is at a space, the latter will be displayed as visible
        % space at end of first line, and a continuation symbol starts next line.
        % Stretch/shrink are however usually zero for typewriter font.
        \def\FV@Space {%
            \nobreak\hskip\z@ plus\fontdimen3\font minus\fontdimen4\font
            \discretionary{\copy\Wrappedvisiblespacebox}{\Wrappedafterbreak}
            {\kern\fontdimen2\font}%
        }%
        
        % Allow breaks at special characters using \PYG... macros.
        \Wrappedbreaksatspecials
        % Breaks at punctuation characters . , ; ? ! and / need catcode=\active 	
        \OriginalVerbatim[#1,codes*=\Wrappedbreaksatpunct]%
    }
    \makeatother

    % Exact colors from NB
    \definecolor{incolor}{HTML}{303F9F}
    \definecolor{outcolor}{HTML}{D84315}
    \definecolor{cellborder}{HTML}{CFCFCF}
    \definecolor{cellbackground}{HTML}{F7F7F7}
    
    % prompt
    \makeatletter
    \newcommand{\boxspacing}{\kern\kvtcb@left@rule\kern\kvtcb@boxsep}
    \makeatother
    \newcommand{\prompt}[4]{
        \ttfamily\llap{{\color{#2}[#3]:\hspace{3pt}#4}}\vspace{-\baselineskip}
    }
    

    
    % Prevent overflowing lines due to hard-to-break entities
    \sloppy 
    % Setup hyperref package
    \hypersetup{
      breaklinks=true,  % so long urls are correctly broken across lines
      colorlinks=true,
      urlcolor=urlcolor,
      linkcolor=linkcolor,
      citecolor=citecolor,
      }
    % Slightly bigger margins than the latex defaults
    
    \geometry{verbose,tmargin=1in,bmargin=1in,lmargin=1in,rmargin=1in}
    
    

\begin{document}
    
    \maketitle
    
    

    
    \hypertarget{taller-7}{%
\section{Taller 7}\label{taller-7}}

Métodos Computacionales para Políticas Públicas - URosario

\textbf{Entrega: viernes 3-abr-2020 11:59 PM}

    \textbf{{[}Monica Alejandra Robayo Gonzalez{]}}

{[}monica.robayo@urosario.edu.co{]}

    \hypertarget{instrucciones}{%
\subsection{Instrucciones:}\label{instrucciones}}

\begin{itemize}
\tightlist
\item
  Guarde una copia de este \emph{Jupyter Notebook} en su computador,
  idealmente en una carpeta destinada al material del curso.
\item
  Modifique el nombre del archivo del \emph{notebook}, agregando al
  final un guión inferior y su nombre y apellido, separados estos
  últimos por otro guión inferior. Por ejemplo, mi \emph{notebook} se
  llamaría: mcpp\_taller7\_santiago\_matallana
\item
  Marque el \emph{notebook} con su nombre y e-mail en el bloque verde
  arriba. Reemplace el texto ``{[}Su nombre acá{]}'' con su nombre y
  apellido. Similar para su e-mail.
\item
  Desarrolle la totalidad del taller sobre este \emph{notebook},
  insertando las celdas que sea necesario debajo de cada pregunta. Haga
  buen uso de las celdas para código y de las celdas tipo
  \emph{markdown} según el caso.
\item
  Recuerde salvar periódicamente sus avances.
\item
  Cuando termine el taller:

  \begin{enumerate}
  \def\labelenumi{\arabic{enumi}.}
  \tightlist
  \item
    Descárguelo en PDF. Si tiene algún problema con la conversión,
    descárguelo en HTML.
  \item
    Suba todos los archivos a su repositorio en GitHub, en una carpeta
    destinada exclusivamente para este taller, antes de la fecha y hora
    límites.
  \end{enumerate}
\end{itemize}

(Todos los ejercicios tienen el mismo valor.)

    \begin{center}\rule{0.5\linewidth}{\linethickness}\end{center}

    Este taller tiene dos partes. Una obligatoria, relativamente fácil, y
otra voluntaria y más retadora. Los invito a intentar desarrollar el
taller en su totalidad. (Buen plan para el aislamiento obligatorio.)

    En este taller exploraremos los datos de crimen de Chicago.

Descargue los datos de crimen del Chicago Data Portal solo para el año
2015
(https://data.cityofchicago.org/Public-Safety/Crimes-2001-to-present/ijzp-q8t2/data).

    \hypertarget{parte-obligatoria}{%
\subsubsection{Parte obligatoria}\label{parte-obligatoria}}

    \begin{tcolorbox}[breakable, size=fbox, boxrule=1pt, pad at break*=1mm,colback=cellbackground, colframe=cellborder]
\prompt{In}{incolor}{1}{\boxspacing}
\begin{Verbatim}[commandchars=\\\{\}]
\PY{k+kn}{import} \PY{n+nn}{pandas} \PY{k}{as} \PY{n+nn}{pd}
\PY{k+kn}{import} \PY{n+nn}{numpy} \PY{k}{as} \PY{n+nn}{np}
\PY{k+kn}{import} \PY{n+nn}{matplotlib}\PY{n+nn}{.}\PY{n+nn}{pyplot} \PY{k}{as} \PY{n+nn}{plt} 
\PY{n}{plt}\PY{o}{.}\PY{n}{rcParams}\PY{p}{[}\PY{l+s+s2}{\PYZdq{}}\PY{l+s+s2}{figure.figsize}\PY{l+s+s2}{\PYZdq{}}\PY{p}{]} \PY{o}{=} \PY{p}{[}\PY{l+m+mf}{18.0}\PY{p}{,} \PY{l+m+mf}{8.0}\PY{p}{]}
\PY{n}{plt}\PY{o}{.}\PY{n}{style}\PY{o}{.}\PY{n}{use}\PY{p}{(}\PY{l+s+s1}{\PYZsq{}}\PY{l+s+s1}{ggplot}\PY{l+s+s1}{\PYZsq{}}\PY{p}{)} \PY{c+c1}{\PYZsh{} R}
\end{Verbatim}
\end{tcolorbox}

    \hypertarget{section}{%
\subsubsection{1.}\label{section}}

Calcule el número de crímenes en cada Community Area en 2015. Haga un
gráfico de barras que lo ilustre.

    \begin{tcolorbox}[breakable, size=fbox, boxrule=1pt, pad at break*=1mm,colback=cellbackground, colframe=cellborder]
\prompt{In}{incolor}{2}{\boxspacing}
\begin{Verbatim}[commandchars=\\\{\}]
\PY{n}{crimes} \PY{o}{=} \PY{n}{pd}\PY{o}{.}\PY{n}{read\PYZus{}csv}\PY{p}{(}\PY{l+s+s2}{\PYZdq{}}\PY{l+s+s2}{Crimes\PYZus{}2015.csv}\PY{l+s+s2}{\PYZdq{}}\PY{p}{)}
\end{Verbatim}
\end{tcolorbox}

    \begin{tcolorbox}[breakable, size=fbox, boxrule=1pt, pad at break*=1mm,colback=cellbackground, colframe=cellborder]
\prompt{In}{incolor}{3}{\boxspacing}
\begin{Verbatim}[commandchars=\\\{\}]
\PY{n}{crimes}
\end{Verbatim}
\end{tcolorbox}

            \begin{tcolorbox}[breakable, size=fbox, boxrule=.5pt, pad at break*=1mm, opacityfill=0]
\prompt{Out}{outcolor}{3}{\boxspacing}
\begin{Verbatim}[commandchars=\\\{\}]
              ID Case Number                    Date                   Block  \textbackslash{}
0       10201852    HY389096  01/01/2015 12:00:00 AM   008XX N MAPLEWOOD AVE
1       10060114    HY239140  01/01/2015 12:00:00 AM     069XX S CORNELL AVE
2       10210454    HY397301  01/01/2015 12:00:00 AM    049XX W WABANSIA AVE
3       10025440    HY214766  01/01/2015 12:00:00 AM         004XX E 80TH ST
4       10225520    HY412735  01/01/2015 12:00:00 AM  075XX S BLACKSTONE AVE
{\ldots}          {\ldots}         {\ldots}                     {\ldots}                     {\ldots}
258508  10369351    HZ105321  12/31/2015 12:00:00 AM     0000X W RANDOLPH ST
258509  10365050    HZ100594  12/31/2015 12:00:00 AM      038XX N FREMONT ST
258510  10364335    HY556149  12/31/2015 12:00:00 AM       037XX W ARGYLE ST
258511  10371800    HZ106801  12/31/2015 12:00:00 AM       019XX N Hoyne Ave
258512  10364402    HY556328  12/31/2015 12:00:00 AM  051XX S BLACKSTONE AVE

        IUCR                Primary Type  \textbackslash{}
0       2825               OTHER OFFENSE
1       1751  OFFENSE INVOLVING CHILDREN
2       0266         CRIM SEXUAL ASSAULT
3       1154          DECEPTIVE PRACTICE
4       1153          DECEPTIVE PRACTICE
{\ldots}      {\ldots}                         {\ldots}
258508  1150          DECEPTIVE PRACTICE
258509  0890                       THEFT
258510  1320             CRIMINAL DAMAGE
258511  1153          DECEPTIVE PRACTICE
258512  0810                       THEFT

                                    Description     Location Description  \textbackslash{}
0                       HARASSMENT BY TELEPHONE                APARTMENT
1                  CRIM SEX ABUSE BY FAM MEMBER                RESIDENCE
2                                     PREDATORY                RESIDENCE
3       FINANCIAL IDENTITY THEFT \$300 AND UNDER                RESIDENCE
4           FINANCIAL IDENTITY THEFT OVER \$ 300                RESIDENCE
{\ldots}                                         {\ldots}                      {\ldots}
258508                        CREDIT CARD FRAUD                    OTHER
258509                            FROM BUILDING  RESIDENCE PORCH/HALLWAY
258510                               TO VEHICLE                   STREET
258511      FINANCIAL IDENTITY THEFT OVER \$ 300                      NaN
258512                                OVER \$500                   STREET

        Arrest  Domestic  {\ldots}  Ward  Community Area  FBI Code  X Coordinate  \textbackslash{}
0        False      True  {\ldots}     1              24        26     1159250.0
1        False     False  {\ldots}     5              43        20     1188551.0
2        False      True  {\ldots}    37              25        02           NaN
3        False      True  {\ldots}     6              44        11     1180776.0
4        False     False  {\ldots}     5              43        11           NaN
{\ldots}        {\ldots}       {\ldots}  {\ldots}   {\ldots}             {\ldots}       {\ldots}           {\ldots}
258508   False     False  {\ldots}    42              32        11     1176135.0
258509   False     False  {\ldots}    46               6        06     1169575.0
258510   False     False  {\ldots}    39              14        14     1150389.0
258511   False     False  {\ldots}    32              22        11     1162044.0
258512   False     False  {\ldots}     4              41        06     1186787.0

       Y Coordinate  Year              Updated On   Latitude  Longitude  \textbackslash{}
0         1905420.0  2015  08/20/2015 04:17:37 PM  41.896198 -87.690553
1         1859284.0  2015  08/17/2015 03:03:40 PM  41.768946 -87.584415
2               NaN  2015  08/31/2015 03:43:09 PM        NaN        NaN
3         1852066.0  2015  08/17/2015 03:03:40 PM  41.749321 -87.613135
4               NaN  2015  09/10/2015 11:43:14 AM        NaN        NaN
{\ldots}             {\ldots}   {\ldots}                     {\ldots}        {\ldots}        {\ldots}
258508    1901285.0  2015  01/15/2016 11:08:45 AM  41.884487 -87.628663
258509    1925758.0  2015  01/07/2016 04:14:34 PM  41.951788 -87.652038
258510    1932961.0  2015  01/07/2016 04:14:34 PM  41.971950 -87.722377
258511    1913082.0  2015  01/15/2016 11:08:45 AM  41.917165 -87.680077
258512    1871326.0  2015  01/07/2016 04:14:34 PM  41.802032 -87.590499

                             Location
0       (41.896197984, -87.690552821)
1       (41.768945532, -87.584414851)
2                                 NaN
3       (41.749320815, -87.613135423)
4                                 NaN
{\ldots}                               {\ldots}
258508  (41.884487451, -87.628662734)
258509  (41.951787958, -87.652037805)
258510  (41.971950353, -87.722376878)
258511  (41.917165184, -87.680076699)
258512   (41.802031713, -87.59049938)

[258513 rows x 22 columns]
\end{Verbatim}
\end{tcolorbox}
        
    \begin{tcolorbox}[breakable, size=fbox, boxrule=1pt, pad at break*=1mm,colback=cellbackground, colframe=cellborder]
\prompt{In}{incolor}{4}{\boxspacing}
\begin{Verbatim}[commandchars=\\\{\}]
\PY{n}{crimes\PYZus{}by\PYZus{}community} \PY{o}{=} \PY{n}{crimes}\PY{o}{.}\PY{n}{groupby}\PY{p}{(}\PY{l+s+s2}{\PYZdq{}}\PY{l+s+s2}{Community Area}\PY{l+s+s2}{\PYZdq{}}\PY{p}{)}
\end{Verbatim}
\end{tcolorbox}

    \begin{tcolorbox}[breakable, size=fbox, boxrule=1pt, pad at break*=1mm,colback=cellbackground, colframe=cellborder]
\prompt{In}{incolor}{5}{\boxspacing}
\begin{Verbatim}[commandchars=\\\{\}]
\PY{n}{community\PYZus{}crime\PYZus{}count} \PY{o}{=} \PY{n}{crimes\PYZus{}by\PYZus{}community}\PY{p}{[}\PY{l+s+s2}{\PYZdq{}}\PY{l+s+s2}{ID}\PY{l+s+s2}{\PYZdq{}}\PY{p}{]}\PY{o}{.}\PY{n}{agg}\PY{p}{(}\PY{l+s+s2}{\PYZdq{}}\PY{l+s+s2}{count}\PY{l+s+s2}{\PYZdq{}}\PY{p}{)}

\PY{n}{community\PYZus{}crime\PYZus{}count}\PY{o}{.}\PY{n}{to\PYZus{}frame}\PY{p}{(}\PY{p}{)}
\end{Verbatim}
\end{tcolorbox}

            \begin{tcolorbox}[breakable, size=fbox, boxrule=.5pt, pad at break*=1mm, opacityfill=0]
\prompt{Out}{outcolor}{5}{\boxspacing}
\begin{Verbatim}[commandchars=\\\{\}]
                  ID
Community Area
1               3519
2               3059
3               3585
4               1747
5               1375
{\ldots}              {\ldots}
73              3109
74               608
75              2052
76              1622
77              2209

[77 rows x 1 columns]
\end{Verbatim}
\end{tcolorbox}
        
    \begin{tcolorbox}[breakable, size=fbox, boxrule=1pt, pad at break*=1mm,colback=cellbackground, colframe=cellborder]
\prompt{In}{incolor}{6}{\boxspacing}
\begin{Verbatim}[commandchars=\\\{\}]
\PY{n}{community\PYZus{}crime\PYZus{}count}\PY{o}{.}\PY{n}{plot}\PY{p}{(}\PY{n}{kind}\PY{o}{=}\PY{l+s+s2}{\PYZdq{}}\PY{l+s+s2}{bar}\PY{l+s+s2}{\PYZdq{}}\PY{p}{,} \PY{n}{color} \PY{o}{=} \PY{l+s+s2}{\PYZdq{}}\PY{l+s+s2}{y}\PY{l+s+s2}{\PYZdq{}}\PY{p}{)}\PY{p}{;}
\end{Verbatim}
\end{tcolorbox}

    \begin{center}
    \adjustimage{max size={0.9\linewidth}{0.9\paperheight}}{output_13_0.png}
    \end{center}
    { \hspace*{\fill} \\}
    
    \begin{center}\rule{0.5\linewidth}{\linethickness}\end{center}

    \hypertarget{section}{%
\subsubsection{2.}\label{section}}

Ordene las Community Areas de acuerdo con el número de crímenes. ¿Qué
Community Area (por nombre, idealmente) presenta el mayor número de
crímenes? ¿El menor?

    \begin{tcolorbox}[breakable, size=fbox, boxrule=1pt, pad at break*=1mm,colback=cellbackground, colframe=cellborder]
\prompt{In}{incolor}{7}{\boxspacing}
\begin{Verbatim}[commandchars=\\\{\}]
\PY{n}{names}\PY{o}{=} \PY{n}{pd}\PY{o}{.}\PY{n}{read\PYZus{}csv}\PY{p}{(}\PY{l+s+s1}{\PYZsq{}}\PY{l+s+s1}{Census\PYZus{}Data\PYZus{}\PYZhy{}\PYZus{}Selected\PYZus{}socioeconomic\PYZus{}indicators\PYZus{}in\PYZus{}Chicago\PYZus{}\PYZus{}2008\PYZus{}\PYZus{}\PYZus{}2012.csv}\PY{l+s+s1}{\PYZsq{}}\PY{p}{)}
\end{Verbatim}
\end{tcolorbox}

    \begin{tcolorbox}[breakable, size=fbox, boxrule=1pt, pad at break*=1mm,colback=cellbackground, colframe=cellborder]
\prompt{In}{incolor}{8}{\boxspacing}
\begin{Verbatim}[commandchars=\\\{\}]
\PY{n}{names}\PY{o}{.}\PY{n}{rename}\PY{p}{(}\PY{n}{columns}\PY{o}{=}\PY{p}{\PYZob{}}\PY{l+s+s2}{\PYZdq{}}\PY{l+s+s2}{Community Area Number}\PY{l+s+s2}{\PYZdq{}}\PY{p}{:} \PY{l+s+s2}{\PYZdq{}}\PY{l+s+s2}{Community Area}\PY{l+s+s2}{\PYZdq{}}\PY{p}{\PYZcb{}}\PY{p}{,} \PY{n}{inplace}\PY{o}{=}\PY{k+kc}{True}\PY{p}{)}
\end{Verbatim}
\end{tcolorbox}

    \begin{tcolorbox}[breakable, size=fbox, boxrule=1pt, pad at break*=1mm,colback=cellbackground, colframe=cellborder]
\prompt{In}{incolor}{9}{\boxspacing}
\begin{Verbatim}[commandchars=\\\{\}]
\PY{n}{names}
\end{Verbatim}
\end{tcolorbox}

            \begin{tcolorbox}[breakable, size=fbox, boxrule=.5pt, pad at break*=1mm, opacityfill=0]
\prompt{Out}{outcolor}{9}{\boxspacing}
\begin{Verbatim}[commandchars=\\\{\}]
    Community Area COMMUNITY AREA NAME  PERCENT OF HOUSING CROWDED  \textbackslash{}
0              1.0         Rogers Park                         7.7
1              2.0          West Ridge                         7.8
2              3.0              Uptown                         3.8
3              4.0      Lincoln Square                         3.4
4              5.0        North Center                         0.3
..             {\ldots}                 {\ldots}                         {\ldots}
73            74.0     Mount Greenwood                         1.0
74            75.0         Morgan Park                         0.8
75            76.0              O'Hare                         3.6
76            77.0           Edgewater                         4.1
77             NaN             CHICAGO                         4.7

    PERCENT HOUSEHOLDS BELOW POVERTY  PERCENT AGED 16+ UNEMPLOYED  \textbackslash{}
0                               23.6                          8.7
1                               17.2                          8.8
2                               24.0                          8.9
3                               10.9                          8.2
4                                7.5                          5.2
..                               {\ldots}                          {\ldots}
73                               3.4                          8.7
74                              13.2                         15.0
75                              15.4                          7.1
76                              18.2                          9.2
77                              19.7                         12.9

    PERCENT AGED 25+ WITHOUT HIGH SCHOOL DIPLOMA  \textbackslash{}
0                                           18.2
1                                           20.8
2                                           11.8
3                                           13.4
4                                            4.5
..                                           {\ldots}
73                                           4.3
74                                          10.8
75                                          10.9
76                                           9.7
77                                          19.5

    PERCENT AGED UNDER 18 OR OVER 64  PER CAPITA INCOME   HARDSHIP INDEX
0                               27.5               23939            39.0
1                               38.5               23040            46.0
2                               22.2               35787            20.0
3                               25.5               37524            17.0
4                               26.2               57123             6.0
..                               {\ldots}                 {\ldots}             {\ldots}
73                              36.8               34381            16.0
74                              40.3               27149            30.0
75                              30.3               25828            24.0
76                              23.8               33385            19.0
77                              33.5               28202             NaN

[78 rows x 9 columns]
\end{Verbatim}
\end{tcolorbox}
        
    \begin{tcolorbox}[breakable, size=fbox, boxrule=1pt, pad at break*=1mm,colback=cellbackground, colframe=cellborder]
\prompt{In}{incolor}{14}{\boxspacing}
\begin{Verbatim}[commandchars=\\\{\}]
\PY{n}{districName}\PY{o}{=}\PY{n}{crimes}\PY{o}{.}\PY{n}{merge}\PY{p}{(}\PY{n}{names}\PY{p}{,}\PY{n}{on}\PY{o}{=}\PY{l+s+s2}{\PYZdq{}}\PY{l+s+s2}{Community Area}\PY{l+s+s2}{\PYZdq{}}\PY{p}{)}
\end{Verbatim}
\end{tcolorbox}

    \begin{tcolorbox}[breakable, size=fbox, boxrule=1pt, pad at break*=1mm,colback=cellbackground, colframe=cellborder]
\prompt{In}{incolor}{15}{\boxspacing}
\begin{Verbatim}[commandchars=\\\{\}]
\PY{n}{crimes\PYZus{}by\PYZus{}community\PYZus{}districName} \PY{o}{=} \PY{n}{districName}\PY{o}{.}\PY{n}{groupby}\PY{p}{(}\PY{l+s+s2}{\PYZdq{}}\PY{l+s+s2}{COMMUNITY AREA NAME}\PY{l+s+s2}{\PYZdq{}}\PY{p}{)}
\end{Verbatim}
\end{tcolorbox}

    \begin{tcolorbox}[breakable, size=fbox, boxrule=1pt, pad at break*=1mm,colback=cellbackground, colframe=cellborder]
\prompt{In}{incolor}{16}{\boxspacing}
\begin{Verbatim}[commandchars=\\\{\}]
\PY{n}{community\PYZus{}crimes\PYZus{}count} \PY{o}{=} \PY{n}{crimes\PYZus{}by\PYZus{}community\PYZus{}districName}\PY{p}{[}\PY{l+s+s1}{\PYZsq{}}\PY{l+s+s1}{ID}\PY{l+s+s1}{\PYZsq{}}\PY{p}{]}\PY{o}{.}\PY{n}{agg}\PY{p}{(}\PY{l+s+s1}{\PYZsq{}}\PY{l+s+s1}{count}\PY{l+s+s1}{\PYZsq{}}\PY{p}{)} 
\PY{n}{community\PYZus{}crimes\PYZus{}count}\PY{o}{.}\PY{n}{to\PYZus{}frame}\PY{p}{(}\PY{p}{)} 
\end{Verbatim}
\end{tcolorbox}

            \begin{tcolorbox}[breakable, size=fbox, boxrule=.5pt, pad at break*=1mm, opacityfill=0]
\prompt{Out}{outcolor}{16}{\boxspacing}
\begin{Verbatim}[commandchars=\\\{\}]
                       ID
COMMUNITY AREA NAME
Albany Park          2501
Archer Heights        986
Armour Square        1057
Ashburn              2279
Auburn Gresham       7733
{\ldots}                   {\ldots}
West Lawn            2027
West Pullman         3949
West Ridge           3059
West Town            6959
Woodlawn             3665

[77 rows x 1 columns]
\end{Verbatim}
\end{tcolorbox}
        
    \begin{tcolorbox}[breakable, size=fbox, boxrule=1pt, pad at break*=1mm,colback=cellbackground, colframe=cellborder]
\prompt{In}{incolor}{17}{\boxspacing}
\begin{Verbatim}[commandchars=\\\{\}]
\PY{n}{community\PYZus{}crimes\PYZus{}count}\PY{o}{.}\PY{n}{to\PYZus{}frame}\PY{p}{(}\PY{p}{)}\PY{o}{.}\PY{n}{sort\PYZus{}values}\PY{p}{(}\PY{n}{by}\PY{o}{=}\PY{l+s+s1}{\PYZsq{}}\PY{l+s+s1}{ID}\PY{l+s+s1}{\PYZsq{}}\PY{p}{,} \PY{n}{ascending}\PY{o}{=}\PY{k+kc}{False}\PY{p}{)}
\end{Verbatim}
\end{tcolorbox}

            \begin{tcolorbox}[breakable, size=fbox, boxrule=.5pt, pad at break*=1mm, opacityfill=0]
\prompt{Out}{outcolor}{17}{\boxspacing}
\begin{Verbatim}[commandchars=\\\{\}]
                        ID
COMMUNITY AREA NAME
Austin               17020
Near North Side       8920
South Shore           8906
North Lawndale        8039
Humboldt park         8015
{\ldots}                    {\ldots}
Montclaire             572
Hegewisch              506
Forest Glen            444
Burnside               380
Edison Park            254

[77 rows x 1 columns]
\end{Verbatim}
\end{tcolorbox}
        
    R/. La Community Area con mayor numero de crimenes es ``AUSTIN'' y la de
menor numero de crimenes es ``EDISON PARK'', para el año 2015.

    \hypertarget{section}{%
\subsubsection{3.}\label{section}}

Cree una tabla cuyas filas sean días del año (yyyy-mm-dd) y las columnas
las 77 Community Areas. En cada campo de la tabla deberá haber el
correspondiente número de crímenes. Seleccione algunas Community Areas
que le llamen la atención y haga un gráfico de serie de tiempo.

Pista: El siguiente código puede serle útil.

    \begin{tcolorbox}[breakable, size=fbox, boxrule=1pt, pad at break*=1mm,colback=cellbackground, colframe=cellborder]
\prompt{In}{incolor}{21}{\boxspacing}
\begin{Verbatim}[commandchars=\\\{\}]
\PY{n}{crimes} \PY{o}{=} \PY{n}{pd}\PY{o}{.}\PY{n}{read\PYZus{}csv}\PY{p}{(}\PY{l+s+s1}{\PYZsq{}}\PY{l+s+s1}{Crimes\PYZus{}2015.csv}\PY{l+s+s1}{\PYZsq{}}\PY{p}{,} \PY{n}{parse\PYZus{}dates}\PY{o}{=}\PY{p}{[}\PY{l+s+s1}{\PYZsq{}}\PY{l+s+s1}{Date}\PY{l+s+s1}{\PYZsq{}}\PY{p}{]}\PY{p}{)}
\end{Verbatim}
\end{tcolorbox}

    \begin{tcolorbox}[breakable, size=fbox, boxrule=1pt, pad at break*=1mm,colback=cellbackground, colframe=cellborder]
\prompt{In}{incolor}{23}{\boxspacing}
\begin{Verbatim}[commandchars=\\\{\}]
\PY{c+c1}{\PYZsh{} Create function to strip time from date field, and use it to create another column}
\PY{k}{def} \PY{n+nf}{to\PYZus{}day}\PY{p}{(}\PY{n}{timestamp}\PY{p}{)}\PY{p}{:}
    \PY{k}{return} \PY{n}{timestamp}\PY{o}{.}\PY{n}{replace}\PY{p}{(}\PY{n}{minute}\PY{o}{=}\PY{l+m+mi}{0}\PY{p}{,}\PY{n}{hour}\PY{o}{=}\PY{l+m+mi}{0}\PY{p}{,} \PY{n}{second}\PY{o}{=}\PY{l+m+mi}{0}\PY{p}{)}

\PY{n}{crimes}\PY{p}{[}\PY{l+s+s2}{\PYZdq{}}\PY{l+s+s2}{Day}\PY{l+s+s2}{\PYZdq{}}\PY{p}{]} \PY{o}{=} \PY{n}{crimes}\PY{p}{[}\PY{l+s+s2}{\PYZdq{}}\PY{l+s+s2}{Date}\PY{l+s+s2}{\PYZdq{}}\PY{p}{]}\PY{o}{.}\PY{n}{apply}\PY{p}{(}\PY{n}{to\PYZus{}day}\PY{p}{)}
\end{Verbatim}
\end{tcolorbox}

    \begin{tcolorbox}[breakable, size=fbox, boxrule=1pt, pad at break*=1mm,colback=cellbackground, colframe=cellborder]
\prompt{In}{incolor}{27}{\boxspacing}
\begin{Verbatim}[commandchars=\\\{\}]
\PY{n}{crimes\PYZus{}by\PYZus{}community\PYZus{}day} \PY{o}{=} \PY{n}{crimes}\PY{o}{.}\PY{n}{groupby}\PY{p}{(}\PY{p}{[}\PY{l+s+s1}{\PYZsq{}}\PY{l+s+s1}{Day}\PY{l+s+s1}{\PYZsq{}}\PY{p}{,}\PY{l+s+s1}{\PYZsq{}}\PY{l+s+s1}{Community Area}\PY{l+s+s1}{\PYZsq{}}\PY{p}{]}\PY{p}{)} 
\PY{n}{crimes\PYZus{}by\PYZus{}community\PYZus{}day\PYZus{}count} \PY{o}{=} \PY{n}{crimes\PYZus{}by\PYZus{}community\PYZus{}day}\PY{p}{[}\PY{l+s+s1}{\PYZsq{}}\PY{l+s+s1}{ID}\PY{l+s+s1}{\PYZsq{}}\PY{p}{]}\PY{o}{.}\PY{n}{agg}\PY{p}{(}\PY{l+s+s1}{\PYZsq{}}\PY{l+s+s1}{count}\PY{l+s+s1}{\PYZsq{}}\PY{p}{)} 
\PY{n}{crimes\PYZus{}by\PYZus{}community\PYZus{}day\PYZus{}count}
\end{Verbatim}
\end{tcolorbox}

            \begin{tcolorbox}[breakable, size=fbox, boxrule=.5pt, pad at break*=1mm, opacityfill=0]
\prompt{Out}{outcolor}{27}{\boxspacing}
\begin{Verbatim}[commandchars=\\\{\}]
Day         Community Area
2015-01-01  1                 13
            2                  7
            3                 11
            4                  4
            5                  5
                              ..
2015-12-31  6                  1
            14                 1
            22                 1
            32                 1
            41                 1
Name: ID, Length: 26803, dtype: int64
\end{Verbatim}
\end{tcolorbox}
        
    \begin{tcolorbox}[breakable, size=fbox, boxrule=1pt, pad at break*=1mm,colback=cellbackground, colframe=cellborder]
\prompt{In}{incolor}{28}{\boxspacing}
\begin{Verbatim}[commandchars=\\\{\}]
\PY{n}{crimes\PYZus{}by\PYZus{}community\PYZus{}day\PYZus{}count}\PY{o}{.}\PY{n}{to\PYZus{}frame}\PY{p}{(}\PY{p}{)}
\end{Verbatim}
\end{tcolorbox}

            \begin{tcolorbox}[breakable, size=fbox, boxrule=.5pt, pad at break*=1mm, opacityfill=0]
\prompt{Out}{outcolor}{28}{\boxspacing}
\begin{Verbatim}[commandchars=\\\{\}]
                           ID
Day        Community Area
2015-01-01 1               13
           2                7
           3               11
           4                4
           5                5
{\ldots}                        ..
2015-12-31 6                1
           14               1
           22               1
           32               1
           41               1

[26803 rows x 1 columns]
\end{Verbatim}
\end{tcolorbox}
        
    \begin{tcolorbox}[breakable, size=fbox, boxrule=1pt, pad at break*=1mm,colback=cellbackground, colframe=cellborder]
\prompt{In}{incolor}{30}{\boxspacing}
\begin{Verbatim}[commandchars=\\\{\}]
\PY{n}{community\PYZus{}crime\PYZus{}byday} \PY{o}{=} \PY{n}{crimes\PYZus{}by\PYZus{}community\PYZus{}day\PYZus{}count}\PY{o}{.}\PY{n}{unstack}\PY{p}{(}\PY{l+s+s1}{\PYZsq{}}\PY{l+s+s1}{Community Area}\PY{l+s+s1}{\PYZsq{}}\PY{p}{)} 
\PY{n}{community\PYZus{}crime\PYZus{}byday}
\end{Verbatim}
\end{tcolorbox}

            \begin{tcolorbox}[breakable, size=fbox, boxrule=.5pt, pad at break*=1mm, opacityfill=0]
\prompt{Out}{outcolor}{30}{\boxspacing}
\begin{Verbatim}[commandchars=\\\{\}]
Community Area    1     2     3     4    5     6     7     8    9    10  {\ldots}  \textbackslash{}
Day                                                                      {\ldots}
2015-01-01      13.0   7.0  11.0   4.0  5.0  22.0  12.0  43.0  1.0  5.0  {\ldots}
2015-01-02       5.0   9.0   8.0   3.0  2.0  10.0   9.0  27.0  NaN  2.0  {\ldots}
2015-01-03       7.0  11.0   9.0   7.0  4.0   6.0  11.0  27.0  1.0  3.0  {\ldots}
2015-01-04      12.0   7.0   9.0  10.0  3.0  15.0   5.0  16.0  1.0  4.0  {\ldots}
2015-01-05       6.0   7.0   5.0   4.0  5.0  15.0   7.0  11.0  1.0  3.0  {\ldots}
{\ldots}              {\ldots}   {\ldots}   {\ldots}   {\ldots}  {\ldots}   {\ldots}   {\ldots}   {\ldots}  {\ldots}  {\ldots}  {\ldots}
2015-12-27      14.0   8.0   6.0   3.0  1.0  16.0  11.0  32.0  NaN  1.0  {\ldots}
2015-12-28       7.0   8.0   5.0   2.0  2.0  10.0   8.0  19.0  NaN  3.0  {\ldots}
2015-12-29       6.0   7.0  12.0   8.0  3.0   8.0   5.0  25.0  NaN  1.0  {\ldots}
2015-12-30       5.0   8.0   7.0   4.0  1.0  11.0  15.0  27.0  1.0  6.0  {\ldots}
2015-12-31       NaN   NaN   NaN   NaN  NaN   1.0   NaN   NaN  NaN  NaN  {\ldots}

Community Area    68    69   70    71   72    73   74   75   76   77
Day
2015-01-01      29.0  23.0  9.0  44.0  2.0   8.0  2.0  5.0  6.0  8.0
2015-01-02      12.0  21.0  5.0  17.0  1.0  11.0  1.0  2.0  6.0  5.0
2015-01-03      23.0  12.0  8.0  18.0  NaN   8.0  1.0  7.0  3.0  3.0
2015-01-04      13.0  15.0  9.0  12.0  1.0   5.0  NaN  1.0  6.0  1.0
2015-01-05      16.0  12.0  8.0  17.0  NaN   5.0  2.0  2.0  7.0  5.0
{\ldots}              {\ldots}   {\ldots}  {\ldots}   {\ldots}  {\ldots}   {\ldots}  {\ldots}  {\ldots}  {\ldots}  {\ldots}
2015-12-27      11.0  19.0  3.0  26.0  2.0   8.0  2.0  1.0  4.0  2.0
2015-12-28      12.0  22.0  9.0  14.0  2.0   6.0  2.0  2.0  3.0  9.0
2015-12-29      18.0  16.0  7.0  18.0  NaN   8.0  3.0  2.0  2.0  4.0
2015-12-30      11.0  23.0  6.0  14.0  2.0   8.0  1.0  7.0  5.0  5.0
2015-12-31       NaN   NaN  NaN   NaN  NaN   NaN  NaN  NaN  NaN  NaN

[365 rows x 77 columns]
\end{Verbatim}
\end{tcolorbox}
        
    \begin{tcolorbox}[breakable, size=fbox, boxrule=1pt, pad at break*=1mm,colback=cellbackground, colframe=cellborder]
\prompt{In}{incolor}{37}{\boxspacing}
\begin{Verbatim}[commandchars=\\\{\}]
\PY{c+c1}{\PYZsh{}Community Areas: las tres primeras Community areas con mayores crimenes: Austin, Near North Side y South Shore}
\PY{n}{community\PYZus{}crime\PYZus{}byday}\PY{p}{[}\PY{p}{[}\PY{l+m+mi}{8}\PY{p}{,}\PY{l+m+mi}{43}\PY{p}{,}\PY{l+m+mi}{25}\PY{p}{]}\PY{p}{]}\PY{o}{.}\PY{n}{plot}\PY{p}{(}\PY{p}{)}\PY{p}{;}
\end{Verbatim}
\end{tcolorbox}

    \begin{center}
    \adjustimage{max size={0.9\linewidth}{0.9\paperheight}}{output_30_0.png}
    \end{center}
    { \hspace*{\fill} \\}
    
    \begin{center}\rule{0.5\linewidth}{\linethickness}\end{center}

    \hypertarget{parte-voluntaria}{%
\subsubsection{Parte voluntaria}\label{parte-voluntaria}}

    Descargue la base de datos de información socioeconómica
(https://data.cityofchicago.org/Health-Human-Services/Census-Data-Selected-socioeconomic-indicators-in-C/kn9c-c2s2).

    \hypertarget{section}{%
\subsubsection{4.}\label{section}}

Cree una tabla que agregue el número de crímenes por Community Area. Una
esa tabla con la de datos socioeconómicos y cree un ``scatter plot'' de
número de crímenes vs ingreso per cápita. Explique la relación en
palabras.

    \begin{tcolorbox}[breakable, size=fbox, boxrule=1pt, pad at break*=1mm,colback=cellbackground, colframe=cellborder]
\prompt{In}{incolor}{112}{\boxspacing}
\begin{Verbatim}[commandchars=\\\{\}]
\PY{n}{crimes\PYZus{}by\PYZus{}community2} \PY{o}{=} \PY{n}{crimes}\PY{o}{.}\PY{n}{groupby}\PY{p}{(}\PY{l+s+s1}{\PYZsq{}}\PY{l+s+s1}{Community Area}\PY{l+s+s1}{\PYZsq{}}\PY{p}{)} 
\PY{n}{community\PYZus{}crimes\PYZus{}count2} \PY{o}{=} \PY{n}{crimes\PYZus{}by\PYZus{}community2}\PY{p}{[}\PY{l+s+s1}{\PYZsq{}}\PY{l+s+s1}{ID}\PY{l+s+s1}{\PYZsq{}}\PY{p}{]}\PY{o}{.}\PY{n}{agg}\PY{p}{(}\PY{l+s+s1}{\PYZsq{}}\PY{l+s+s1}{count}\PY{l+s+s1}{\PYZsq{}}\PY{p}{)} 
\PY{n}{table\PYZus{}crimes\PYZus{}by\PYZus{}community\PYZus{}area}\PY{o}{=}\PY{n}{community\PYZus{}crimes\PYZus{}count2}\PY{o}{.}\PY{n}{to\PYZus{}frame}\PY{p}{(}\PY{p}{)}
\end{Verbatim}
\end{tcolorbox}

    \begin{tcolorbox}[breakable, size=fbox, boxrule=1pt, pad at break*=1mm,colback=cellbackground, colframe=cellborder]
\prompt{In}{incolor}{113}{\boxspacing}
\begin{Verbatim}[commandchars=\\\{\}]
\PY{c+c1}{\PYZsh{} cree la tabla de crimenes por community area}
\PY{n}{table\PYZus{}crimes\PYZus{}by\PYZus{}community\PYZus{}area} 
\end{Verbatim}
\end{tcolorbox}

            \begin{tcolorbox}[breakable, size=fbox, boxrule=.5pt, pad at break*=1mm, opacityfill=0]
\prompt{Out}{outcolor}{113}{\boxspacing}
\begin{Verbatim}[commandchars=\\\{\}]
                  ID
Community Area
1               3519
2               3059
3               3585
4               1747
5               1375
{\ldots}              {\ldots}
73              3109
74               608
75              2052
76              1622
77              2209

[77 rows x 1 columns]
\end{Verbatim}
\end{tcolorbox}
        
    \begin{tcolorbox}[breakable, size=fbox, boxrule=1pt, pad at break*=1mm,colback=cellbackground, colframe=cellborder]
\prompt{In}{incolor}{114}{\boxspacing}
\begin{Verbatim}[commandchars=\\\{\}]
\PY{c+c1}{\PYZsh{}adjunte la base de datos como percapita}
\PY{n}{percapita}\PY{o}{=} \PY{n}{pd}\PY{o}{.}\PY{n}{read\PYZus{}csv}\PY{p}{(}\PY{l+s+s1}{\PYZsq{}}\PY{l+s+s1}{Census\PYZus{}Data\PYZus{}\PYZhy{}\PYZus{}Selected\PYZus{}socioeconomic\PYZus{}indicators\PYZus{}in\PYZus{}Chicago\PYZus{}\PYZus{}2008\PYZus{}\PYZus{}\PYZus{}2012.csv}\PY{l+s+s1}{\PYZsq{}}\PY{p}{)} 
\end{Verbatim}
\end{tcolorbox}

    \begin{tcolorbox}[breakable, size=fbox, boxrule=1pt, pad at break*=1mm,colback=cellbackground, colframe=cellborder]
\prompt{In}{incolor}{115}{\boxspacing}
\begin{Verbatim}[commandchars=\\\{\}]
\PY{c+c1}{\PYZsh{}cambie el nombre de la columna \PYZdq{}Community Area Number\PYZdq{} por \PYZdq{}Community Area\PYZdq{}}
\PY{n}{percapita}\PY{o}{.}\PY{n}{rename}\PY{p}{(}\PY{n}{columns}\PY{o}{=}\PY{p}{\PYZob{}}\PY{l+s+s2}{\PYZdq{}}\PY{l+s+s2}{Community Area Number}\PY{l+s+s2}{\PYZdq{}}\PY{p}{:} \PY{l+s+s2}{\PYZdq{}}\PY{l+s+s2}{Community Area}\PY{l+s+s2}{\PYZdq{}}\PY{p}{\PYZcb{}}\PY{p}{,} \PY{n}{inplace}\PY{o}{=}\PY{k+kc}{True}\PY{p}{)}
\end{Verbatim}
\end{tcolorbox}

    \begin{tcolorbox}[breakable, size=fbox, boxrule=1pt, pad at break*=1mm,colback=cellbackground, colframe=cellborder]
\prompt{In}{incolor}{116}{\boxspacing}
\begin{Verbatim}[commandchars=\\\{\}]
\PY{n}{percapita} 
\end{Verbatim}
\end{tcolorbox}

            \begin{tcolorbox}[breakable, size=fbox, boxrule=.5pt, pad at break*=1mm, opacityfill=0]
\prompt{Out}{outcolor}{116}{\boxspacing}
\begin{Verbatim}[commandchars=\\\{\}]
    Community Area COMMUNITY AREA NAME  PERCENT OF HOUSING CROWDED  \textbackslash{}
0              1.0         Rogers Park                         7.7
1              2.0          West Ridge                         7.8
2              3.0              Uptown                         3.8
3              4.0      Lincoln Square                         3.4
4              5.0        North Center                         0.3
..             {\ldots}                 {\ldots}                         {\ldots}
73            74.0     Mount Greenwood                         1.0
74            75.0         Morgan Park                         0.8
75            76.0              O'Hare                         3.6
76            77.0           Edgewater                         4.1
77             NaN             CHICAGO                         4.7

    PERCENT HOUSEHOLDS BELOW POVERTY  PERCENT AGED 16+ UNEMPLOYED  \textbackslash{}
0                               23.6                          8.7
1                               17.2                          8.8
2                               24.0                          8.9
3                               10.9                          8.2
4                                7.5                          5.2
..                               {\ldots}                          {\ldots}
73                               3.4                          8.7
74                              13.2                         15.0
75                              15.4                          7.1
76                              18.2                          9.2
77                              19.7                         12.9

    PERCENT AGED 25+ WITHOUT HIGH SCHOOL DIPLOMA  \textbackslash{}
0                                           18.2
1                                           20.8
2                                           11.8
3                                           13.4
4                                            4.5
..                                           {\ldots}
73                                           4.3
74                                          10.8
75                                          10.9
76                                           9.7
77                                          19.5

    PERCENT AGED UNDER 18 OR OVER 64  PER CAPITA INCOME   HARDSHIP INDEX
0                               27.5               23939            39.0
1                               38.5               23040            46.0
2                               22.2               35787            20.0
3                               25.5               37524            17.0
4                               26.2               57123             6.0
..                               {\ldots}                 {\ldots}             {\ldots}
73                              36.8               34381            16.0
74                              40.3               27149            30.0
75                              30.3               25828            24.0
76                              23.8               33385            19.0
77                              33.5               28202             NaN

[78 rows x 9 columns]
\end{Verbatim}
\end{tcolorbox}
        
    \begin{tcolorbox}[breakable, size=fbox, boxrule=1pt, pad at break*=1mm,colback=cellbackground, colframe=cellborder]
\prompt{In}{incolor}{117}{\boxspacing}
\begin{Verbatim}[commandchars=\\\{\}]
\PY{c+c1}{\PYZsh{}para unir la tabla a la base datos \PYZdq{}percapita\PYZdq{} utilizo \PYZdq{}pd.merge\PYZdq{}}
\PY{n}{pib\PYZus{}percapita\PYZus{}income} \PY{o}{=} \PY{n}{pd}\PY{o}{.}\PY{n}{merge}\PY{p}{(}\PY{n}{percapita}\PY{p}{,} \PY{n}{table\PYZus{}crimes\PYZus{}by\PYZus{}community\PYZus{}area}\PY{p}{,} \PY{n}{on}\PY{o}{=}\PY{l+s+s1}{\PYZsq{}}\PY{l+s+s1}{Community Area}\PY{l+s+s1}{\PYZsq{}}\PY{p}{)} 
\end{Verbatim}
\end{tcolorbox}

    \begin{tcolorbox}[breakable, size=fbox, boxrule=1pt, pad at break*=1mm,colback=cellbackground, colframe=cellborder]
\prompt{In}{incolor}{118}{\boxspacing}
\begin{Verbatim}[commandchars=\\\{\}]
\PY{c+c1}{\PYZsh{} al ejecutar el codigo se evidencia la columna \PYZdq{}ID\PYZdq{}}
\PY{n}{pib\PYZus{}percapita\PYZus{}income}
\end{Verbatim}
\end{tcolorbox}

            \begin{tcolorbox}[breakable, size=fbox, boxrule=.5pt, pad at break*=1mm, opacityfill=0]
\prompt{Out}{outcolor}{118}{\boxspacing}
\begin{Verbatim}[commandchars=\\\{\}]
    Community Area COMMUNITY AREA NAME  PERCENT OF HOUSING CROWDED  \textbackslash{}
0              1.0         Rogers Park                         7.7
1              2.0          West Ridge                         7.8
2              3.0              Uptown                         3.8
3              4.0      Lincoln Square                         3.4
4              5.0        North Center                         0.3
..             {\ldots}                 {\ldots}                         {\ldots}
72            73.0   Washington Height                         1.1
73            74.0     Mount Greenwood                         1.0
74            75.0         Morgan Park                         0.8
75            76.0              O'Hare                         3.6
76            77.0           Edgewater                         4.1

    PERCENT HOUSEHOLDS BELOW POVERTY  PERCENT AGED 16+ UNEMPLOYED  \textbackslash{}
0                               23.6                          8.7
1                               17.2                          8.8
2                               24.0                          8.9
3                               10.9                          8.2
4                                7.5                          5.2
..                               {\ldots}                          {\ldots}
72                              16.9                         20.8
73                               3.4                          8.7
74                              13.2                         15.0
75                              15.4                          7.1
76                              18.2                          9.2

    PERCENT AGED 25+ WITHOUT HIGH SCHOOL DIPLOMA  \textbackslash{}
0                                           18.2
1                                           20.8
2                                           11.8
3                                           13.4
4                                            4.5
..                                           {\ldots}
72                                          13.7
73                                           4.3
74                                          10.8
75                                          10.9
76                                           9.7

    PERCENT AGED UNDER 18 OR OVER 64  PER CAPITA INCOME   HARDSHIP INDEX    ID
0                               27.5               23939            39.0  3519
1                               38.5               23040            46.0  3059
2                               22.2               35787            20.0  3585
3                               25.5               37524            17.0  1747
4                               26.2               57123             6.0  1375
..                               {\ldots}                 {\ldots}             {\ldots}   {\ldots}
72                              42.6               19713            48.0  3109
73                              36.8               34381            16.0   608
74                              40.3               27149            30.0  2052
75                              30.3               25828            24.0  1622
76                              23.8               33385            19.0  2209

[77 rows x 10 columns]
\end{Verbatim}
\end{tcolorbox}
        
    \begin{tcolorbox}[breakable, size=fbox, boxrule=1pt, pad at break*=1mm,colback=cellbackground, colframe=cellborder]
\prompt{In}{incolor}{119}{\boxspacing}
\begin{Verbatim}[commandchars=\\\{\}]
\PY{c+c1}{\PYZsh{} aqui confirmo la existencia de la columna \PYZdq{}ID\PYZdq{}}
\PY{n}{pib\PYZus{}percapita\PYZus{}income}\PY{o}{.}\PY{n}{columns}
\end{Verbatim}
\end{tcolorbox}

            \begin{tcolorbox}[breakable, size=fbox, boxrule=.5pt, pad at break*=1mm, opacityfill=0]
\prompt{Out}{outcolor}{119}{\boxspacing}
\begin{Verbatim}[commandchars=\\\{\}]
Index(['Community Area', 'COMMUNITY AREA NAME', 'PERCENT OF HOUSING CROWDED',
       'PERCENT HOUSEHOLDS BELOW POVERTY', 'PERCENT AGED 16+ UNEMPLOYED',
       'PERCENT AGED 25+ WITHOUT HIGH SCHOOL DIPLOMA',
       'PERCENT AGED UNDER 18 OR OVER 64', 'PER CAPITA INCOME ',
       'HARDSHIP INDEX', 'ID'],
      dtype='object')
\end{Verbatim}
\end{tcolorbox}
        
    \begin{tcolorbox}[breakable, size=fbox, boxrule=1pt, pad at break*=1mm,colback=cellbackground, colframe=cellborder]
\prompt{In}{incolor}{135}{\boxspacing}
\begin{Verbatim}[commandchars=\\\{\}]
\PY{c+c1}{\PYZsh{}realizo una indexacion con la columna \PYZdq{}PER CAPITA INCOME\PYZdq{} y \PYZdq{}ID\PYZdq{} a la cual llamé \PYZdq{}crimes\PYZus{}vs\PYZus{}percapita\PYZus{}income\PYZdq{}}
\PY{n}{crimes\PYZus{}vs\PYZus{}percapita\PYZus{}income} \PY{o}{=} \PY{n}{pib\PYZus{}percapita\PYZus{}income}\PY{o}{.}\PY{n}{loc}\PY{p}{[}\PY{l+m+mi}{1}\PY{p}{:}\PY{l+m+mi}{78}\PY{p}{,}\PY{p}{[}\PY{l+s+s1}{\PYZsq{}}\PY{l+s+s1}{PER CAPITA INCOME }\PY{l+s+s1}{\PYZsq{}}\PY{p}{,}\PY{l+s+s1}{\PYZsq{}}\PY{l+s+s1}{ID}\PY{l+s+s1}{\PYZsq{}}\PY{p}{]}\PY{p}{]}
\end{Verbatim}
\end{tcolorbox}

    \begin{tcolorbox}[breakable, size=fbox, boxrule=1pt, pad at break*=1mm,colback=cellbackground, colframe=cellborder]
\prompt{In}{incolor}{136}{\boxspacing}
\begin{Verbatim}[commandchars=\\\{\}]
\PY{c+c1}{\PYZsh{}al ejecutar el codigo se evidencia \PYZdq{}crimes\PYZus{}vs\PYZus{}percapita\PYZus{}income\PYZdq{}}
\PY{n}{crimes\PYZus{}vs\PYZus{}percapita\PYZus{}income}
\end{Verbatim}
\end{tcolorbox}

            \begin{tcolorbox}[breakable, size=fbox, boxrule=.5pt, pad at break*=1mm, opacityfill=0]
\prompt{Out}{outcolor}{136}{\boxspacing}
\begin{Verbatim}[commandchars=\\\{\}]
    PER CAPITA INCOME     ID
1                23040  3059
2                35787  3585
3                37524  1747
4                57123  1375
5                60058  5495
..                 {\ldots}   {\ldots}
72               19713  3109
73               34381   608
74               27149  2052
75               25828  1622
76               33385  2209

[76 rows x 2 columns]
\end{Verbatim}
\end{tcolorbox}
        
    \begin{tcolorbox}[breakable, size=fbox, boxrule=1pt, pad at break*=1mm,colback=cellbackground, colframe=cellborder]
\prompt{In}{incolor}{140}{\boxspacing}
\begin{Verbatim}[commandchars=\\\{\}]
\PY{c+c1}{\PYZsh{}realizo el scatter plot con \PYZdq{}crimes\PYZus{}vs\PYZus{}percapita\PYZus{}income\PYZdq{}}
\PY{n}{my\PYZus{}plot} \PY{o}{=} \PY{n}{crimes\PYZus{}vs\PYZus{}percapita\PYZus{}income}\PY{o}{.}\PY{n}{plot}\PY{p}{(}\PY{l+s+s1}{\PYZsq{}}\PY{l+s+s1}{ID}\PY{l+s+s1}{\PYZsq{}}\PY{p}{,} \PY{l+s+s1}{\PYZsq{}}\PY{l+s+s1}{PER CAPITA INCOME }\PY{l+s+s1}{\PYZsq{}}\PY{p}{,} \PY{n}{kind}\PY{o}{=}\PY{l+s+s2}{\PYZdq{}}\PY{l+s+s2}{scatter}\PY{l+s+s2}{\PYZdq{}}\PY{p}{)}
\PY{n}{plt}\PY{o}{.}\PY{n}{show}\PY{p}{(}\PY{p}{)}
\end{Verbatim}
\end{tcolorbox}

    \begin{center}
    \adjustimage{max size={0.9\linewidth}{0.9\paperheight}}{output_45_0.png}
    \end{center}
    { \hspace*{\fill} \\}
    
    Comentario: se evidencia que la mayor concentración de crimenes se da en
aquellas areas donde el ingreso per capita oscila entre 10.000 y 30.000
dolares, es decir en aquella población donde sus ingresos en promedio no
son relativamente altos; y a medida que aumenta el ingreso percapita
disminuye la concentración de crimenes registrados.

Con ello, superficialmente se puede evidenciar que a mayor ingreso
percapita, menor numero de concentracion de crimenes se registran, y a
menor ingreso percapita se registra una mayor concentración de numero de
crimenes. Sin embargo, no es prudente afirmar la existencia de alguna
relación perfecta entrega ambas variables, toda vez que probablemente
existan otras variables que pueden explicar el efecto causal.

    \begin{center}\rule{0.5\linewidth}{\linethickness}\end{center}


    % Add a bibliography block to the postdoc
    
    
    
\end{document}
